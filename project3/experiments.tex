%%%%%%%%%%%%%%%%%%%%%%%%%%%%%%%%%%%%%%%%%%%%%%%%%%%%%%%%%%%%%%%%%%%%%%
% LaTeX Example: Project Report
%
% Source: http://www.howtotex.com
%
% Feel free to distribute this example, but please keep the referral
% to howtotex.com
% Date: March 2011 
% 
%%%%%%%%%%%%%%%%%%%%%%%%%%%%%%%%%%%%%%%%%%%%%%%%%%%%%%%%%%%%%%%%%%%%%%
% How to use writeLaTeX: 
%
% You edit the source code here on the left, and the preview on the
% right shows you the result within a few seconds.
%
% Bookmark this page and share the URL with your co-authors. They can
% edit at the same time!
%
% You can upload figures, bibliographies, custom classes and
% styles using the files menu.
%
% If you're new to LaTeX, the wikibook is a great place to start:
% http://en.wikibooks.org/wiki/LaTeX
%
%%%%%%%%%%%%%%%%%%%%%%%%%%%%%%%%%%%%%%%%%%%%%%%%%%%%%%%%%%%%%%%%%%%%%%
% Edit the title below to update the display in My Documents
%\title{Project Report}
%
%%% Preamble
\documentclass[paper=a4, fontsize=11pt]{scrartcl}
\usepackage[T1]{fontenc}
\usepackage{fourier}

\usepackage[english]{babel}															% English language/hyphenation
\usepackage[protrusion=true,expansion=true]{microtype}	
\usepackage{amsmath,amsfonts,amsthm} % Math packages
\usepackage[pdftex]{graphicx}	
\usepackage{url}


%%% Custom sectioning
\usepackage{sectsty}
\allsectionsfont{\centering \normalfont\scshape}


%%% Custom headers/footers (fancyhdr package)
\usepackage{fancyhdr}
\pagestyle{fancyplain}
\fancyhead{}											% No page header
\fancyfoot[L]{}											% Empty 
\fancyfoot[C]{}											% Empty
\fancyfoot[R]{\thepage}									% Pagenumbering
\renewcommand{\headrulewidth}{0pt}			% Remove header underlines
\renewcommand{\footrulewidth}{0pt}				% Remove footer underlines
\setlength{\headheight}{13.6pt}

%%% Equation and float numbering
\numberwithin{equation}{section}		% Equationnumbering: section.eq#
\numberwithin{figure}{section}			% Figurenumbering: section.fig#
\numberwithin{table}{section}				% Tablenumbering: section.tab#


%%% Maketitle metadata
\newcommand{\horrule}[1]{\rule{\linewidth}{#1}} 	% Horizontal rule

\title{
		%\vspace{-1in} 	
		\usefont{OT1}{bch}{b}{n}
		\normalfont \normalsize \textsc{School of random department names} \\ [25pt]
		\horrule{0.5pt} \\[0.4cm]
		\huge This is the title of the template report \\
		\horrule{2pt} \\[0.5cm]
}
\author{
		\normalfont 								\normalsize
        Firstname Lastname\\[-3pt]		\normalsize
        \today
}
\date{}


%%% Begin document
\begin{document}
\section{Experiments}
\subsection{Expectations}
When comparing the 4 different queue implementations, we expect them to perform as follow:
\begin{description}
\item[1. Haskell List]
The Haskell List queue implementation performs \texttt{inject} in $O(n)$ time, and as such, we expect it to perform horribly on any test case that requires pushes.
On pure \texttt{pop} operations however, it will be $O(1)$, with a very low overhead, and as such it will most likely be among the fastest queues at performing \texttt{pop}s-
\item[2. Paired O(1) Non-Reusable Queue]
The Paired Queue implementation should be fairly low-overhead, and as such perform quite well, except on the special test case, where we reuse queues.
\item[3. Real-Time Strict Queues] 
The Real-Time Strict Queue should generally perform well on all test cases, since we have a nice $O(1)$ worst-case guarentee for all operations. This queue however, has quite a large overhead, and will therefore most likely be outperformed by most other queues in tests cases where they exhibit $O(1)$ performance. 
\item[4. Lazy Amortized $O(1)$ Queues]
The Lazy Queue should perform very well on all test cases. Not only does it have $O(1)$ amortized cost for all operation, it also has quite a low overhead. 
\end{description}
\end{document}
