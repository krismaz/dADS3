\section{Binary Heap running times}

\subsection{MakeHeap}

Constructing and empty Binary Heap requires only a constant amount of work, if we look at the heap as a graph of nodes, Since we simply need to specify that the size is 0, and that there is no root. As such, this operation will be $O(1)$.

Note that an array-backed, Binary Heap will need to initialize an array of size $O(n)$, which might not be not be a constant-time operation, depending on the underlying hardware and operating system.

\subsection{FindMin}

Finding the minimum in an Binary Heap is a simple matter of locating the root of the heap. This will either be stored as a pointer in the heap, or, in the case of an array-backed heap, at the start of the array. As such, the operation is $O(1)$.

\subsection{Insert}

Inserting an element in a Binary Heap is done by inserting the element at the first free position at the bottom of the heap, and repeatedly swapping it with its parent until it is at the correct position. Since we perform only a constant amount of operations at each level of the heap, and the heap is logarithmic in height, this operation take $O(\log (n))$ time.

\subsection{DeleteMin}

Deleting the minimum element of a Binary Heap is done by swapping the root and the last element of the heap, removing the last element (the previous root), and swapping the new root with its largest child until it is at the correct position int the heap. Much like the Insert operation, we perform only a constant amount of work for each level of the heap, and this makes the operation take $O(\log (n))$ time.

\subsection{DecreaseKey}

Decreasing a key in a Binary Heap can be done by decreasing the key value of the specified node, and repeatetly swapping it with its parents until it is at the correct position in the heap for its new value. Since this is the same complexity as an insert, it takes $O(\log (n))$ time.


\section{Fibonacci Heap running times}

\subsection{MakeHeap}

Initializing a Fibonacci heap is a matter of setting a few variables to their default values, such as setting the size to zero, the root list to an empty list and no value for the minimum node. This can be done in $O(1)$ time.

\subsection{FindMin}

A Fibonacci Heap keeps a pointer to the minimum element at all times. This makes finding the minimum element an $O(1)$ operation.

\subsection{Insert}

Inserting an element into a Fibonacci Heap is done is done by creating a new node, and appending it to the root list. If the root list is a data structure that supports inserts in constant time, this can be done in $O(1)$ time.

\subsection{DeleteMin}

Deleting the minimum of a Fibonacci Heap is done by adding all of the minimum nodes children to the root list, removing the minimum, and then continously linking root nodes until only root node of each degree exists. After this, we look through the root list, and set the minimum to be the root with the lowest key.

This leads to the following time requirements. 

\begin{description}
\item[Finding the minimum] As shown earlier, $O(1)$.
\item[Adding all the children to root list] Since the number of children to a node is limited by its degree, and the maximum degree of a node is logarithmic with respect to $n$, we can do this in $O(\log (n))$ time as long as we can insert elements into our root list in constant time.
\item[Linking roots] Since linking two root nodes together takes time, and the linking of two roots removes one of them from the root list, we have that we can link all of the roots in $O(\#roots)$ time. Since we can have only one root for each element in the heap, this can take $O(n)$ time.
\item[Finding the new minimum] After we have linked the roots together, we have at most one root for each degree, and since the maximum degree of a root if logarithmic, we have that we can search through the new list of roots in $O(\log (n))$ time.
\end{description}

This gives us a total worst case running time of the DeleteMin operation of $O(n)$ time.

\subsection{DecreaseKey}

Decreasing the key of a node in a Fibonacci Heap, is done by decreasing the key of the node, and removing it from the child list of the parent, and adding it to the root list, we then continue to do this for the parent as long as we keep encountering marked parents, and mark the final, non-marked, parent we encounter.

In the worst case this requires us to traverse all the way up to the root of the tree our node belonged to. Note that the height of a tree is limited by the degree of its root, which is at most logarithmic. This means that the operation takes at most $O(\log (n))$ time.