\section{Introduction}

Heaps are a very useful kind of data structure, and several algorithms depend one them. As seen in the previous project, choosing the right heap for the job at hand can prove to be quite an improvement in running times.

In this project, we will look at Van Emde Boas Trees, and how they can be used to create a quite efficient heap. Van Emde Boas trees are quite peculiar data structures, whose upper bound for the \texttt{Insert}, \texttt{Delete} \texttt{Member} and \texttt{Predecessor} are $O(\log ( \log (U)))$, with $U$ being the universe size. However, they have they drawback of not supporting  using $O(U)$ memory, unless we have access to perfect hashing.

In this project we hope to show that by limiting the key type of the heap to be 24-bit integers and not supporting the \texttt{DecreaseKey} operation, we can construct a very efficient heap backed by a Van Emde Boas tree. Also, we wish to compare the performance of Van Emde Boas trees to more general trees supporting the \texttt{Predessecor} operation, specifically Red-Black trees, and hope to show that the Van Emde Boas trees are superior even on these simple operations.